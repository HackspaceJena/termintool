\documentclass[ngerman]{paper}
%\renewcommand{\familydefault}{\ttdefault}
\usepackage[T1]{fontenc}
\usepackage[utf8]{inputenc}
\usepackage{geometry}
\geometry{verbose,tmargin=0cm,bmargin=2cm,lmargin=2cm,rmargin=2cm}
\usepackage{array}
\usepackage{booktabs}
\usepackage{palatino}
\usepackage[T1]{fontenc}
\usepackage[utf8]{inputenc}
\usepackage[draft=false]{microtype}
\makeatletter
\providecommand{\tabularnewline}{\\}

\makeatother

\usepackage{babel}
\begin{document}

\title{Veranstaltungen im Hackspace KW 3 /2013}

\maketitle

\section*{Idee und Motivation}

Ein Hackspace ist ein Platz an dem man sich trifft, um gemeinsam an
verschiedenen Themen rund um Technik zu arbeiten. Dies nicht zum Gelderwerb,
sondern aus Spaß an der Sache.


\section*{Wie gewohnt das Kulturprogramm für kommende Woche:}

\begin{tabular*}{\columnwidth}{@{\extracolsep{\fill}}>{\raggedright}p{0.28\columnwidth}>{\raggedright}p{0.72\columnwidth}}
{{#events}}
{{startdate}}: & \textbf{\large  {{&heading}} \normalsize}

{{&text}}\tabularnewline
\midrule
{{/events}}
\end{tabular*}


\section*{Regelmäßig im Space:}

\small
\begin{itemize}
	\item \textbf{montags:} Elektronikrunde ab ca. 19:30h
	\item \textbf{dienstags:} offene Runde ab ca. 20h
	\item \textbf{mittwochs:} gerade Wochen: (Brett-)spieleabend
	\item \textbf{donnerstags:} ungerade Wochen: Stammtisch der Linux-Nutzer-Gruppe
	\item \textbf{freitags:} \begin{itemize}
		\item 1. Woche im Monat: Lockpicking ab ca. 20h
		\item 2. Woche im Monat: Plenum ab ca. 19h
		\item 3. Woche im Monat: Küchenecke ab ca. 19h
	\end{itemize}
	\item \textbf{sonntags:} ungerade Wochen: Chaoscafe ab ca. 16h
\end{itemize}


\section*{Kontakt}
\begin{description}
\item [{Physisch}] in unserem Raum \textquotedbl{}Krautspace\textquotedbl{}
in der Krautgasse 26
\item [{per~Mail}] auf unserer Mailingliste https://www.krautspace.de/hswiki:mailingliste
\item [{oder~in~unserem~Chat}] (Jabber MUC): hackspace@chat.lug-jena.de
(Webinterface)\end{description}
\vfill
\raggedleft{\textbf{Stand:} \today}
\end{document}
